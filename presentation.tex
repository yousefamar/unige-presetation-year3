\documentclass[aspectratio=169]{beamer}
\mode<presentation>

\hypersetup{pdfpagemode=FullScreen}
\useoutertheme[subsection=false]{smoothbars}

\usepackage[utf8]{inputenc}
\usepackage{tikz}
\usetikzlibrary{shapes}
\usetikzlibrary{arrows.meta}
\usepackage{listings}
\graphicspath{{figs/}}

\colorlet{punct}{red!60!black}
\definecolor{delim}{RGB}{20,105,176}

\lstdefinelanguage{json}{%
	basicstyle=\scriptsize\ttfamily,
	showstringspaces=false,
	breaklines=true,
	literate=
		*{:}{{{\color{punct}{:}}}}{1}
		{,}{{{\color{punct}{,}}}}{1}
		{\{}{{{\color{delim}{\{}}}}{1}
		{\}}{{{\color{delim}{\}}}}}{1}
		{[}{{{\color{delim}{[}}}}{1}
		{]}{{{\color{delim}{]}}}}{1}
}

\lstdefinelanguage{dockerfile}{%
	basicstyle=\ttfamily,
	showstringspaces=false,
	breaklines=true
}
\lstset{language=dockerfile}

\lstdefinelanguage{caveats}{%
	basicstyle=\scriptsize\ttfamily,
	showstringspaces=false,
	breaklines=true
}

\tikzset{%
	arrow/.style={%
		draw,
		fill=structure.fg,
		single arrow,
		minimum width=5ex,
		minimum height=10ex,
		single arrow head extend=1ex
	}
}
\newcommand{\arrow}{%
\tikz [baseline=-0.5ex]{\node [arrow] {};}
}

\title{\Huge Privacy-Aware Load Balancing for Distributed Computation}
\author{\Large Yousef Amar\\%
\small QMUL Supervisor: Gareth Tyson\\%
UniGe Supervisor: Lucio Marcenaro\\%
Starting Date: 2016-01-16%
}
\date{}
\setbeamertemplate{navigation symbols}{}

\begin{document}

\frame{\titlepage}

\begin{frame}
	\frametitle{Previous Work}
	\begin{itemize}
		\item Privacy-Aware Access Control
		\item Load Balancing on the Network Edge
	\end{itemize}
	\includegraphics[width=\linewidth]{big-picture-plus}
\end{frame}

\begin{frame}
	\frametitle{Research Context}
	\begin{columns}[c]

		\column{.5\textwidth}
		\begin{itemize}
			\item Data sources at the edge and in the cloud
			\begin{itemize}
				\item IoT devices
				\item Smart phones
				\item Social media
			\end{itemize}
			\item Computation shifting towards the edge
			\begin{itemize}
				\item Untapped processing power
				\item Near sources, low latency
				\item An increase in privacy
			\end{itemize}
		\end{itemize}

		\column{.5\textwidth}
		\includegraphics[width=\linewidth]{context}

	\end{columns}
\end{frame}

\begin{frame}
	\frametitle{Research Context}
	\includegraphics[width=\linewidth]{multiplicity}
\end{frame}

\begin{frame}
	\frametitle{Research Problem}
	\begin{itemize}
		\item Which host should execute a given function?
		\item Minimising factors such as
		\begin{itemize}
			\item Latency
			\item Information disclosure
			\item Cost (e.g.\ power consumption, bandwidth usage)
			\item Device capabilities
			\item Other
		\end{itemize}
		\item Tradeoff between these factors
		\item A system for tracking these factors and controlling job distribution
	\end{itemize}
\end{frame}

\begin{frame}
	\frametitle{Related Work}
	\begin{itemize}
		\item SOTA Load Balancing applicable to this context:
		\begin{itemize}
			\item Beamer (NSDI 2018)
			\item Maglev (NSDI 2016)
		\end{itemize}
		\item Lots of research in privacy and information disclosure
		\item \emph{None} combining the load balancing with privacy
	\end{itemize}
\end{frame}

\begin{frame}
	\frametitle{Requirements and Constraints}
	\begin{itemize}
		\item A system that distributes computation in a way that minimises response time, within the bounds of privacy constraints imposed by the user
		\item Overhead must be small enough for execution on edge and home IoT devices
		\item System must be versatile enough to run on a range of edge devices
	\end{itemize}
\end{frame}

\begin{frame}
	\frametitle{Design}
	\framesubtitle{Architechture --- Symbols}
	\begin{columns}[c]
		\column{.5\textwidth}
		\includegraphics[width=\linewidth]{symbols}
		\column{.5\textwidth}
		\includegraphics[width=\linewidth]{privapprox}
	\end{columns}
\end{frame}

\begin{frame}
	\frametitle{Method}
	\framesubtitle{No sources, no cross-host comms}
	\includegraphics[width=\linewidth]{method1}
\end{frame}

\begin{frame}
	\frametitle{Method}
	\framesubtitle{Single-source, no cross-host comms, no linkage}
	\includegraphics[width=\linewidth]{method2a}
\end{frame}

\begin{frame}
	\frametitle{Method}
	\framesubtitle{Single-source, no cross-host comms, no linkage}
	\includegraphics[width=\linewidth]{method2b}
\end{frame}

\begin{frame}
	\frametitle{Method}
	\framesubtitle{Single-source, cross-source comms, host whitelist}
	\includegraphics[width=\linewidth]{method3}
\end{frame}

\begin{frame}
	\frametitle{Method}
	\framesubtitle{Private sources allow larger whitelists}
	\includegraphics[width=\linewidth]{method4}
\end{frame}

\begin{frame}
	\frametitle{Method}
	\framesubtitle{Multi-source, cross-host sources}
	\includegraphics[width=\linewidth]{method5}
\end{frame}

\begin{frame}
	\frametitle{Method}
	\framesubtitle{Whitelist Generation}
	\begin{itemize}
		\item User, community, and/or app provider list inferences for individual and combinations of data source access
		\item System lists risks vs utility, and user decides which risks are acceptable for the utlity they require
		\item System selects a subset of hosts that would satisfy these constraints
		\item From the system-side, classes of jobs simply have a subset of hosts available to execute on
		\item Problem reduces to whitelist generation on top of load balancing
	\end{itemize}
\end{frame}

\begin{frame}
	\frametitle{Method}
	\framesubtitle{Architechture -- Job, data, and metadata pipelines}
	\centering
	\includegraphics[width=0.6\linewidth]{arch-new}
\end{frame}

\begin{frame}
	\frametitle{Method}
	\framesubtitle{Load Balancing -- Optimising for Performance}
	\centering
	\includegraphics[width=0.7\columnwidth]{service}
\end{frame}

\begin{frame}
	\frametitle{Method}
	\framesubtitle{Load Balancing -- Optimising for Performance}
	\begin{columns}[c]
		\column{.6\textwidth}
		Message response times with different partitioning strategies as Tukey Box-Whisker plots. Whiskers show 1.5$\times$ inter-quartile range above (respectively below) the third (respectively first) quartile, whereas red diamonds show the mean values.
		\column{.4\textwidth}
		\includegraphics[width=\columnwidth]{kafka-violins}
	\end{columns}
\end{frame}

\begin{frame}
	\frametitle{Privacy Heuristics}
	\includegraphics[width=0.48\linewidth]{metrics-surprisal}
	\includegraphics[width=0.5\linewidth]{metrics-autocorrelation}
\end{frame}

\begin{frame}
	\frametitle{Past Evaluation}
	\begin{figure}
		\centering
		\includegraphics[width=0.49\columnwidth]{redd-roc-utility-interesting-bit}
		\includegraphics[width=0.49\columnwidth]{redd-roc-attack-interesting-bit}
		\caption{Receiver Operating Characteristic (ROC) curves for washer-dryer (utility; left) and microwave (attack; right)}\label{fig:redd-roc-interesting-bit}
	\end{figure}
\end{frame}

\begin{frame}
	\frametitle{Past Evaluation}
	\begin{columns}[c]

		\column{.45\textwidth}
		\begin{itemize}
			\item Gains in privacy
			\item Without impacting utility
			\item Negligible latency overhead
			\item Showing that the overhead on IoT devices is justifiable
		\end{itemize}

		\column{.55\textwidth}
		\begin{figure}
			\centering
			\includegraphics[width=\columnwidth]{violin}
			\caption{Distributions of time to availability under different conditions}\label{fig:violin}
		\end{figure}

	\end{columns}
\end{frame}

\begin{frame}
	\frametitle{Theis ToC}
	\begin{itemize}
		\item Chapter 1: Privacy Measures
		\item Chapter 2: Privacy-Aware Access Control
		\item Chapter 3: Edge Load Balancing
		\item Chapter 4: Privacy-Aware Edge Load Balancing
	\end{itemize}
\end{frame}

\begin{frame}
	\frametitle{Publications}
	\footnotesize
	\begin{itemize}
		\item 2018 -- Providing Occupancy as a Service with Databox \textit{In Proceedings of the 1st ACM International Workshop on Smart Cities and Fog Computing} (pp.\ 29-34) ACM
		\item 2018 -- An Information-Theoretic Approach to Time-Series Data Privacy \textit{In Proceedings of the 1st Workshop on Privacy by Design in Distributed Systems} (p. 3) ACM
		\item 2018 -- Building accountability into the Internet of Things: the IoT Databox model \textit{Journal of Reliable Intelligent Environments} (pp.\ 1-17) Springer
		\item 2017 -- Balanced Message Distribution in Distributed Message Handling Systems \textit{US Patent (serial number: 15/794440)}
		\item 2017 -- Route-based authorization and discovery for personal data \textit{In the 11th EuroSys Doctoral Workshop}
		\item 2016 -- Personal Data Management with the Databox: What's Inside the Box? \textit{In Proceedings of the 2016 ACM Workshop on Cloud-Assisted Networking} (pp.\ 49-54) ACM
		\item 2016 -- Privacy-aware infrastructure for managing personal data \textit{In Proceedings of the 2016 ACM SIGCOMM Conference} (pp.\ 571-572) ACM
		\item 2015 -- Incremental dense multi-modal 3d scene reconstruction. \textit{In Intelligent Robots and Systems (IROS), 2015 IEEE/RSJ International Conference on} (pp.\ 908-915) IEEE
	\end{itemize}
\end{frame}

\begin{frame}
	\frametitle{Conference and Workshop Attendance}
	\begin{itemize}
		\item W-P2DS 2018 (presenting)
		\item EuroDW 2018 (presenting)
		\item Databox Annual Symposium (presenting)
		\item EuroDW 2017
		\item EuroSys 2017
		\item SIGCOMM 2016
		\item NGN-MSN Coseners 2016
		\item UK HDAN Research Roadmap Workshop
	\end{itemize}
\end{frame}

\begin{frame}
	\frametitle{Awards and Scientific Recognition}
	\begin{itemize}
		\item EuroSys 2018 Conference Grant
		\item EuroSys 2018 Shadow PC Grant
		\item EuroSys 2017 Grant
		\item IEEE Circuit Building Competition Winner
		\item SIGCOMM 2016 Grant
		\item \texttt{yousefamar.com\#awards}
	\end{itemize}
\end{frame}

\begin{frame}
	\frametitle{Miscellaneous}
	\begin{itemize}
		\item amar.io/skills-points-record.png
		\item Involvement in Databox Project databoxproject.uk
		\item Internship at Nokia Bell Labs Stuttgart
		\item Time at UniGe and areas of overlap
	\end{itemize}
\end{frame}

\begin{frame}[plain,c]
	\begin{center}
		\usebeamerfont*{frametitle}
		\usebeamercolor[fg]{frametitle}
		\ \\[2em]
		\Huge Thank you for your attention!\\[1em]
	\end{center}
\end{frame}

\end{document}
